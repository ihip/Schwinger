\documentclass[english]{beamer}
\usetheme{Frankfurt}

\title{Finite temperature and $\delta$-regime in the Schwinger model}
\author{
  Ivan Hip\textsuperscript{a},
  Jaime Fabián Nieto Castellanos\textsuperscript{b},
  Wolfgang Bietenholz\textsuperscript{b}}
\institute{
  \textsuperscript{a}University of Zagreb, Croatia\\
  \textsuperscript{b}UNAM, Mexico
}
\date{July 29, 2021}

\begin{document}
 
\begin{frame}
  \titlepage
\end{frame}

\section{Introduction}

\begin{frame}{Schwinger model}
  \begin{itemize}
    \item two-dimensional quantum electrodynamics - fermions coupled to Abelian gauge field U(1)
    \item introduced by Schwinger in 1961
    \item the simplest example of chiral anomaly, one of the simplest models which illustrates confinement
    \item often used as a testbed for conceptual and numerical
approaches in lattice field theory
    \item nevertheless, some of the rich physical
properties of the model in anisotropic volumes have not yet been tested
    \begin{itemize}
      \item \textbf{finite temperature Schwinger model}: Hosotani solution has not been compared with the lattice simulation results
      \item \textbf{$\delta$-regime}: Hasenfratz and Niedermayer predictions for residual pion mass have not been investigated in the Schwinger model
    \end{itemize}
  \end{itemize}
\end{frame}

\section{Finite temperature}

\begin{frame}{Finite temperature - Hosotani solution}
  \includegraphics[width=1\textwidth]{figs/FiniteTMPiHos}
\end{frame}

\begin{frame}{Pion mass - Hosotani vs. lattice simulation}
  \includegraphics[width=1\textwidth]{figs/MPi64x10FiniteT}
\end{frame}

\begin{frame}{Pion mass - Hosotani vs. lattice simulation}
  \includegraphics[width=1\textwidth]{figs/Meta64x10FiniteT}
\end{frame}

\section{$\delta$-regime}

\begin{frame}{$\delta$-regime}
\end{frame}

\begin{frame}{Hasenfratz/Niedermayer prediction}

\end{frame}

\begin{frame}{Residual pion mass plateau}
  \includegraphics[width=1\textwidth]{figs/Mpi10x64}
\end{frame}

\begin{frame}{Residual pion mass plateau}
  \includegraphics[width=1\textwidth]{figs/Mpi6x64Pt10}
\end{frame}

\begin{frame}{1 / L confirmed by lattice simulation}
  \begin{columns}[t]
    \column{0.5\textwidth}
      \includegraphics[width=1.0\textwidth]{figs/ResMpiBeta2}
      \includegraphics[width=1.0\textwidth]{figs/ResMpiBeta4}
    \column{0.5\textwidth}
      \includegraphics[width=1.0\textwidth]{figs/ResMpiBeta3}
      \begin{center} 
	    \begin{tabular}{c c c c}
	      $\beta$ & $F_\pi$ \\
	      \hline
	      2.0 & 0.6683(50) \\
	      \hline
	      3.0 & 0.6681(25) \\
	      \hline
	      4.0 & 0.6700(22) \\
	    \end{tabular}
        \[
      	  F_\pi = 0.6688(5)
        \]
      \end{center}	 
  \end{columns}
\end{frame}


\section{Witten-Veneziano}

\begin{frame}{Witten-Veneziano formula}
  \begin{itemize}
    \item in the chiral $N$-flavor Schwinger model the Witten-
      Veneziano formula is simplified to [Seiler and Stamatescu,
      1987]
      \[
        m_\eta^2 = \frac{2N}{F_\eta^2}\chi_T^{que}
      \]
    \item mass of the $\eta$ particle is known analytically
      [Belvedere et al. 1979] 
      \[
        m_\eta^2 = \frac{N}{\pi\beta}
      \]
    \item there is also continuum prediction for $\chi_T^{que}$
      [Seiler and Stamatescu, 1987]
      \[
        \beta\chi_T^{que} = \frac{1}{4\pi^2}
      \]
  \end{itemize}
\end{frame}

\begin{frame}{Quenched topological susceptibility}
  \begin{itemize}
    \item {[Bardeen et al., 1998]} were able to analytically
      compute $\chi_T^{que}$ on the lattice
      \[
        \beta\chi_T^{que} = \frac{I_1(\beta)}{4 \pi^2 I_0(\beta)}
      \]
      by using an alternative definition of topological charge
      \[
        Q_S = \frac{1}{2\pi}\sum_{P}\sin(\theta_P)
      \]
    \item for the usual definition of topological charge
      \[
        Q_T = \frac{1}{2\pi}\sum_{P}\theta_P
      \]
      it is not possible to find analytic solution, but using the 
      same line of reasoning it is possible to numerically compute
      $\chi_T^{que}$ to arbitrary precision
  \end{itemize}
\end{frame}

\begin{frame}{Quenched topological susceptibility}
  \includegraphics[width=1\textwidth]{figs/BeakDiagram}
\end{frame}

\begin{frame}{$F_\eta$ versus $F_\pi$}
\begin{itemize}
  \item in large $N_c$ QCD, to the order $1/N_c$
    \[
      F_{\eta'} = F_\pi
    \]
  \item in the Schwinger model nothing assures that this relation holds
  \item inserting the confirmed values for $m_\eta^2$ and $\chi_T^{que}$ 
    \[
      F_{\eta}^2 = \frac{2N}{m_\eta^2}\chi_T^{que} =
        2N \left(\frac{\pi\beta}{N}\right)
        \left(\frac{1}{4\pi^2\beta}\right) =
        \frac{1}{2\pi}
    \]
  \item our results suggest that in the Schwinger model these two decay constants differ significantly
    \[
      F_{\eta} = 0.3989 \qquad F_\pi = 0.6688(5)
    \]
\end{itemize}
\end{frame}

\end{document}
